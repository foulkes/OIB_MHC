
% Default to the notebook output style

    


% Inherit from the specified cell style.




    
\documentclass[11pt]{article}

    
    
    \usepackage[T1]{fontenc}
    % Nicer default font (+ math font) than Computer Modern for most use cases
    \usepackage{mathpazo}

    % Basic figure setup, for now with no caption control since it's done
    % automatically by Pandoc (which extracts ![](path) syntax from Markdown).
    \usepackage{graphicx}
    % We will generate all images so they have a width \maxwidth. This means
    % that they will get their normal width if they fit onto the page, but
    % are scaled down if they would overflow the margins.
    \makeatletter
    \def\maxwidth{\ifdim\Gin@nat@width>\linewidth\linewidth
    \else\Gin@nat@width\fi}
    \makeatother
    \let\Oldincludegraphics\includegraphics
    % Set max figure width to be 80% of text width, for now hardcoded.
    \renewcommand{\includegraphics}[1]{\Oldincludegraphics[width=.8\maxwidth]{#1}}
    % Ensure that by default, figures have no caption (until we provide a
    % proper Figure object with a Caption API and a way to capture that
    % in the conversion process - todo).
    \usepackage{caption}
    \DeclareCaptionLabelFormat{nolabel}{}
    \captionsetup{labelformat=nolabel}

    \usepackage{adjustbox} % Used to constrain images to a maximum size 
    \usepackage{xcolor} % Allow colors to be defined
    \usepackage{enumerate} % Needed for markdown enumerations to work
    \usepackage{geometry} % Used to adjust the document margins
    \usepackage{amsmath} % Equations
    \usepackage{amssymb} % Equations
    \usepackage{textcomp} % defines textquotesingle
    % Hack from http://tex.stackexchange.com/a/47451/13684:
    \AtBeginDocument{%
        \def\PYZsq{\textquotesingle}% Upright quotes in Pygmentized code
    }
    \usepackage{upquote} % Upright quotes for verbatim code
    \usepackage{eurosym} % defines \euro
    \usepackage[mathletters]{ucs} % Extended unicode (utf-8) support
    \usepackage[utf8x]{inputenc} % Allow utf-8 characters in the tex document
    \usepackage{fancyvrb} % verbatim replacement that allows latex
    \usepackage{grffile} % extends the file name processing of package graphics 
                         % to support a larger range 
    % The hyperref package gives us a pdf with properly built
    % internal navigation ('pdf bookmarks' for the table of contents,
    % internal cross-reference links, web links for URLs, etc.)
    \usepackage{hyperref}
    \usepackage{longtable} % longtable support required by pandoc >1.10
    \usepackage{booktabs}  % table support for pandoc > 1.12.2
    \usepackage[inline]{enumitem} % IRkernel/repr support (it uses the enumerate* environment)
    \usepackage[normalem]{ulem} % ulem is needed to support strikethroughs (\sout)
                                % normalem makes italics be italics, not underlines
    

    
    
    % Colors for the hyperref package
    \definecolor{urlcolor}{rgb}{0,.145,.698}
    \definecolor{linkcolor}{rgb}{.71,0.21,0.01}
    \definecolor{citecolor}{rgb}{.12,.54,.11}

    % ANSI colors
    \definecolor{ansi-black}{HTML}{3E424D}
    \definecolor{ansi-black-intense}{HTML}{282C36}
    \definecolor{ansi-red}{HTML}{E75C58}
    \definecolor{ansi-red-intense}{HTML}{B22B31}
    \definecolor{ansi-green}{HTML}{00A250}
    \definecolor{ansi-green-intense}{HTML}{007427}
    \definecolor{ansi-yellow}{HTML}{DDB62B}
    \definecolor{ansi-yellow-intense}{HTML}{B27D12}
    \definecolor{ansi-blue}{HTML}{208FFB}
    \definecolor{ansi-blue-intense}{HTML}{0065CA}
    \definecolor{ansi-magenta}{HTML}{D160C4}
    \definecolor{ansi-magenta-intense}{HTML}{A03196}
    \definecolor{ansi-cyan}{HTML}{60C6C8}
    \definecolor{ansi-cyan-intense}{HTML}{258F8F}
    \definecolor{ansi-white}{HTML}{C5C1B4}
    \definecolor{ansi-white-intense}{HTML}{A1A6B2}

    % commands and environments needed by pandoc snippets
    % extracted from the output of `pandoc -s`
    \providecommand{\tightlist}{%
      \setlength{\itemsep}{0pt}\setlength{\parskip}{0pt}}
    \DefineVerbatimEnvironment{Highlighting}{Verbatim}{commandchars=\\\{\}}
    % Add ',fontsize=\small' for more characters per line
    \newenvironment{Shaded}{}{}
    \newcommand{\KeywordTok}[1]{\textcolor[rgb]{0.00,0.44,0.13}{\textbf{{#1}}}}
    \newcommand{\DataTypeTok}[1]{\textcolor[rgb]{0.56,0.13,0.00}{{#1}}}
    \newcommand{\DecValTok}[1]{\textcolor[rgb]{0.25,0.63,0.44}{{#1}}}
    \newcommand{\BaseNTok}[1]{\textcolor[rgb]{0.25,0.63,0.44}{{#1}}}
    \newcommand{\FloatTok}[1]{\textcolor[rgb]{0.25,0.63,0.44}{{#1}}}
    \newcommand{\CharTok}[1]{\textcolor[rgb]{0.25,0.44,0.63}{{#1}}}
    \newcommand{\StringTok}[1]{\textcolor[rgb]{0.25,0.44,0.63}{{#1}}}
    \newcommand{\CommentTok}[1]{\textcolor[rgb]{0.38,0.63,0.69}{\textit{{#1}}}}
    \newcommand{\OtherTok}[1]{\textcolor[rgb]{0.00,0.44,0.13}{{#1}}}
    \newcommand{\AlertTok}[1]{\textcolor[rgb]{1.00,0.00,0.00}{\textbf{{#1}}}}
    \newcommand{\FunctionTok}[1]{\textcolor[rgb]{0.02,0.16,0.49}{{#1}}}
    \newcommand{\RegionMarkerTok}[1]{{#1}}
    \newcommand{\ErrorTok}[1]{\textcolor[rgb]{1.00,0.00,0.00}{\textbf{{#1}}}}
    \newcommand{\NormalTok}[1]{{#1}}
    
    % Additional commands for more recent versions of Pandoc
    \newcommand{\ConstantTok}[1]{\textcolor[rgb]{0.53,0.00,0.00}{{#1}}}
    \newcommand{\SpecialCharTok}[1]{\textcolor[rgb]{0.25,0.44,0.63}{{#1}}}
    \newcommand{\VerbatimStringTok}[1]{\textcolor[rgb]{0.25,0.44,0.63}{{#1}}}
    \newcommand{\SpecialStringTok}[1]{\textcolor[rgb]{0.73,0.40,0.53}{{#1}}}
    \newcommand{\ImportTok}[1]{{#1}}
    \newcommand{\DocumentationTok}[1]{\textcolor[rgb]{0.73,0.13,0.13}{\textit{{#1}}}}
    \newcommand{\AnnotationTok}[1]{\textcolor[rgb]{0.38,0.63,0.69}{\textbf{\textit{{#1}}}}}
    \newcommand{\CommentVarTok}[1]{\textcolor[rgb]{0.38,0.63,0.69}{\textbf{\textit{{#1}}}}}
    \newcommand{\VariableTok}[1]{\textcolor[rgb]{0.10,0.09,0.49}{{#1}}}
    \newcommand{\ControlFlowTok}[1]{\textcolor[rgb]{0.00,0.44,0.13}{\textbf{{#1}}}}
    \newcommand{\OperatorTok}[1]{\textcolor[rgb]{0.40,0.40,0.40}{{#1}}}
    \newcommand{\BuiltInTok}[1]{{#1}}
    \newcommand{\ExtensionTok}[1]{{#1}}
    \newcommand{\PreprocessorTok}[1]{\textcolor[rgb]{0.74,0.48,0.00}{{#1}}}
    \newcommand{\AttributeTok}[1]{\textcolor[rgb]{0.49,0.56,0.16}{{#1}}}
    \newcommand{\InformationTok}[1]{\textcolor[rgb]{0.38,0.63,0.69}{\textbf{\textit{{#1}}}}}
    \newcommand{\WarningTok}[1]{\textcolor[rgb]{0.38,0.63,0.69}{\textbf{\textit{{#1}}}}}
    
    
    % Define a nice break command that doesn't care if a line doesn't already
    % exist.
    \def\br{\hspace*{\fill} \\* }
    % Math Jax compatability definitions
    \def\gt{>}
    \def\lt{<}
    % Document parameters
    \title{Unit 3 - Exercise (Clinical Trial Simulation) Solutions}
    
    
    

    % Pygments definitions
    
\makeatletter
\def\PY@reset{\let\PY@it=\relax \let\PY@bf=\relax%
    \let\PY@ul=\relax \let\PY@tc=\relax%
    \let\PY@bc=\relax \let\PY@ff=\relax}
\def\PY@tok#1{\csname PY@tok@#1\endcsname}
\def\PY@toks#1+{\ifx\relax#1\empty\else%
    \PY@tok{#1}\expandafter\PY@toks\fi}
\def\PY@do#1{\PY@bc{\PY@tc{\PY@ul{%
    \PY@it{\PY@bf{\PY@ff{#1}}}}}}}
\def\PY#1#2{\PY@reset\PY@toks#1+\relax+\PY@do{#2}}

\expandafter\def\csname PY@tok@w\endcsname{\def\PY@tc##1{\textcolor[rgb]{0.73,0.73,0.73}{##1}}}
\expandafter\def\csname PY@tok@c\endcsname{\let\PY@it=\textit\def\PY@tc##1{\textcolor[rgb]{0.25,0.50,0.50}{##1}}}
\expandafter\def\csname PY@tok@cp\endcsname{\def\PY@tc##1{\textcolor[rgb]{0.74,0.48,0.00}{##1}}}
\expandafter\def\csname PY@tok@k\endcsname{\let\PY@bf=\textbf\def\PY@tc##1{\textcolor[rgb]{0.00,0.50,0.00}{##1}}}
\expandafter\def\csname PY@tok@kp\endcsname{\def\PY@tc##1{\textcolor[rgb]{0.00,0.50,0.00}{##1}}}
\expandafter\def\csname PY@tok@kt\endcsname{\def\PY@tc##1{\textcolor[rgb]{0.69,0.00,0.25}{##1}}}
\expandafter\def\csname PY@tok@o\endcsname{\def\PY@tc##1{\textcolor[rgb]{0.40,0.40,0.40}{##1}}}
\expandafter\def\csname PY@tok@ow\endcsname{\let\PY@bf=\textbf\def\PY@tc##1{\textcolor[rgb]{0.67,0.13,1.00}{##1}}}
\expandafter\def\csname PY@tok@nb\endcsname{\def\PY@tc##1{\textcolor[rgb]{0.00,0.50,0.00}{##1}}}
\expandafter\def\csname PY@tok@nf\endcsname{\def\PY@tc##1{\textcolor[rgb]{0.00,0.00,1.00}{##1}}}
\expandafter\def\csname PY@tok@nc\endcsname{\let\PY@bf=\textbf\def\PY@tc##1{\textcolor[rgb]{0.00,0.00,1.00}{##1}}}
\expandafter\def\csname PY@tok@nn\endcsname{\let\PY@bf=\textbf\def\PY@tc##1{\textcolor[rgb]{0.00,0.00,1.00}{##1}}}
\expandafter\def\csname PY@tok@ne\endcsname{\let\PY@bf=\textbf\def\PY@tc##1{\textcolor[rgb]{0.82,0.25,0.23}{##1}}}
\expandafter\def\csname PY@tok@nv\endcsname{\def\PY@tc##1{\textcolor[rgb]{0.10,0.09,0.49}{##1}}}
\expandafter\def\csname PY@tok@no\endcsname{\def\PY@tc##1{\textcolor[rgb]{0.53,0.00,0.00}{##1}}}
\expandafter\def\csname PY@tok@nl\endcsname{\def\PY@tc##1{\textcolor[rgb]{0.63,0.63,0.00}{##1}}}
\expandafter\def\csname PY@tok@ni\endcsname{\let\PY@bf=\textbf\def\PY@tc##1{\textcolor[rgb]{0.60,0.60,0.60}{##1}}}
\expandafter\def\csname PY@tok@na\endcsname{\def\PY@tc##1{\textcolor[rgb]{0.49,0.56,0.16}{##1}}}
\expandafter\def\csname PY@tok@nt\endcsname{\let\PY@bf=\textbf\def\PY@tc##1{\textcolor[rgb]{0.00,0.50,0.00}{##1}}}
\expandafter\def\csname PY@tok@nd\endcsname{\def\PY@tc##1{\textcolor[rgb]{0.67,0.13,1.00}{##1}}}
\expandafter\def\csname PY@tok@s\endcsname{\def\PY@tc##1{\textcolor[rgb]{0.73,0.13,0.13}{##1}}}
\expandafter\def\csname PY@tok@sd\endcsname{\let\PY@it=\textit\def\PY@tc##1{\textcolor[rgb]{0.73,0.13,0.13}{##1}}}
\expandafter\def\csname PY@tok@si\endcsname{\let\PY@bf=\textbf\def\PY@tc##1{\textcolor[rgb]{0.73,0.40,0.53}{##1}}}
\expandafter\def\csname PY@tok@se\endcsname{\let\PY@bf=\textbf\def\PY@tc##1{\textcolor[rgb]{0.73,0.40,0.13}{##1}}}
\expandafter\def\csname PY@tok@sr\endcsname{\def\PY@tc##1{\textcolor[rgb]{0.73,0.40,0.53}{##1}}}
\expandafter\def\csname PY@tok@ss\endcsname{\def\PY@tc##1{\textcolor[rgb]{0.10,0.09,0.49}{##1}}}
\expandafter\def\csname PY@tok@sx\endcsname{\def\PY@tc##1{\textcolor[rgb]{0.00,0.50,0.00}{##1}}}
\expandafter\def\csname PY@tok@m\endcsname{\def\PY@tc##1{\textcolor[rgb]{0.40,0.40,0.40}{##1}}}
\expandafter\def\csname PY@tok@gh\endcsname{\let\PY@bf=\textbf\def\PY@tc##1{\textcolor[rgb]{0.00,0.00,0.50}{##1}}}
\expandafter\def\csname PY@tok@gu\endcsname{\let\PY@bf=\textbf\def\PY@tc##1{\textcolor[rgb]{0.50,0.00,0.50}{##1}}}
\expandafter\def\csname PY@tok@gd\endcsname{\def\PY@tc##1{\textcolor[rgb]{0.63,0.00,0.00}{##1}}}
\expandafter\def\csname PY@tok@gi\endcsname{\def\PY@tc##1{\textcolor[rgb]{0.00,0.63,0.00}{##1}}}
\expandafter\def\csname PY@tok@gr\endcsname{\def\PY@tc##1{\textcolor[rgb]{1.00,0.00,0.00}{##1}}}
\expandafter\def\csname PY@tok@ge\endcsname{\let\PY@it=\textit}
\expandafter\def\csname PY@tok@gs\endcsname{\let\PY@bf=\textbf}
\expandafter\def\csname PY@tok@gp\endcsname{\let\PY@bf=\textbf\def\PY@tc##1{\textcolor[rgb]{0.00,0.00,0.50}{##1}}}
\expandafter\def\csname PY@tok@go\endcsname{\def\PY@tc##1{\textcolor[rgb]{0.53,0.53,0.53}{##1}}}
\expandafter\def\csname PY@tok@gt\endcsname{\def\PY@tc##1{\textcolor[rgb]{0.00,0.27,0.87}{##1}}}
\expandafter\def\csname PY@tok@err\endcsname{\def\PY@bc##1{\setlength{\fboxsep}{0pt}\fcolorbox[rgb]{1.00,0.00,0.00}{1,1,1}{\strut ##1}}}
\expandafter\def\csname PY@tok@kc\endcsname{\let\PY@bf=\textbf\def\PY@tc##1{\textcolor[rgb]{0.00,0.50,0.00}{##1}}}
\expandafter\def\csname PY@tok@kd\endcsname{\let\PY@bf=\textbf\def\PY@tc##1{\textcolor[rgb]{0.00,0.50,0.00}{##1}}}
\expandafter\def\csname PY@tok@kn\endcsname{\let\PY@bf=\textbf\def\PY@tc##1{\textcolor[rgb]{0.00,0.50,0.00}{##1}}}
\expandafter\def\csname PY@tok@kr\endcsname{\let\PY@bf=\textbf\def\PY@tc##1{\textcolor[rgb]{0.00,0.50,0.00}{##1}}}
\expandafter\def\csname PY@tok@bp\endcsname{\def\PY@tc##1{\textcolor[rgb]{0.00,0.50,0.00}{##1}}}
\expandafter\def\csname PY@tok@fm\endcsname{\def\PY@tc##1{\textcolor[rgb]{0.00,0.00,1.00}{##1}}}
\expandafter\def\csname PY@tok@vc\endcsname{\def\PY@tc##1{\textcolor[rgb]{0.10,0.09,0.49}{##1}}}
\expandafter\def\csname PY@tok@vg\endcsname{\def\PY@tc##1{\textcolor[rgb]{0.10,0.09,0.49}{##1}}}
\expandafter\def\csname PY@tok@vi\endcsname{\def\PY@tc##1{\textcolor[rgb]{0.10,0.09,0.49}{##1}}}
\expandafter\def\csname PY@tok@vm\endcsname{\def\PY@tc##1{\textcolor[rgb]{0.10,0.09,0.49}{##1}}}
\expandafter\def\csname PY@tok@sa\endcsname{\def\PY@tc##1{\textcolor[rgb]{0.73,0.13,0.13}{##1}}}
\expandafter\def\csname PY@tok@sb\endcsname{\def\PY@tc##1{\textcolor[rgb]{0.73,0.13,0.13}{##1}}}
\expandafter\def\csname PY@tok@sc\endcsname{\def\PY@tc##1{\textcolor[rgb]{0.73,0.13,0.13}{##1}}}
\expandafter\def\csname PY@tok@dl\endcsname{\def\PY@tc##1{\textcolor[rgb]{0.73,0.13,0.13}{##1}}}
\expandafter\def\csname PY@tok@s2\endcsname{\def\PY@tc##1{\textcolor[rgb]{0.73,0.13,0.13}{##1}}}
\expandafter\def\csname PY@tok@sh\endcsname{\def\PY@tc##1{\textcolor[rgb]{0.73,0.13,0.13}{##1}}}
\expandafter\def\csname PY@tok@s1\endcsname{\def\PY@tc##1{\textcolor[rgb]{0.73,0.13,0.13}{##1}}}
\expandafter\def\csname PY@tok@mb\endcsname{\def\PY@tc##1{\textcolor[rgb]{0.40,0.40,0.40}{##1}}}
\expandafter\def\csname PY@tok@mf\endcsname{\def\PY@tc##1{\textcolor[rgb]{0.40,0.40,0.40}{##1}}}
\expandafter\def\csname PY@tok@mh\endcsname{\def\PY@tc##1{\textcolor[rgb]{0.40,0.40,0.40}{##1}}}
\expandafter\def\csname PY@tok@mi\endcsname{\def\PY@tc##1{\textcolor[rgb]{0.40,0.40,0.40}{##1}}}
\expandafter\def\csname PY@tok@il\endcsname{\def\PY@tc##1{\textcolor[rgb]{0.40,0.40,0.40}{##1}}}
\expandafter\def\csname PY@tok@mo\endcsname{\def\PY@tc##1{\textcolor[rgb]{0.40,0.40,0.40}{##1}}}
\expandafter\def\csname PY@tok@ch\endcsname{\let\PY@it=\textit\def\PY@tc##1{\textcolor[rgb]{0.25,0.50,0.50}{##1}}}
\expandafter\def\csname PY@tok@cm\endcsname{\let\PY@it=\textit\def\PY@tc##1{\textcolor[rgb]{0.25,0.50,0.50}{##1}}}
\expandafter\def\csname PY@tok@cpf\endcsname{\let\PY@it=\textit\def\PY@tc##1{\textcolor[rgb]{0.25,0.50,0.50}{##1}}}
\expandafter\def\csname PY@tok@c1\endcsname{\let\PY@it=\textit\def\PY@tc##1{\textcolor[rgb]{0.25,0.50,0.50}{##1}}}
\expandafter\def\csname PY@tok@cs\endcsname{\let\PY@it=\textit\def\PY@tc##1{\textcolor[rgb]{0.25,0.50,0.50}{##1}}}

\def\PYZbs{\char`\\}
\def\PYZus{\char`\_}
\def\PYZob{\char`\{}
\def\PYZcb{\char`\}}
\def\PYZca{\char`\^}
\def\PYZam{\char`\&}
\def\PYZlt{\char`\<}
\def\PYZgt{\char`\>}
\def\PYZsh{\char`\#}
\def\PYZpc{\char`\%}
\def\PYZdl{\char`\$}
\def\PYZhy{\char`\-}
\def\PYZsq{\char`\'}
\def\PYZdq{\char`\"}
\def\PYZti{\char`\~}
% for compatibility with earlier versions
\def\PYZat{@}
\def\PYZlb{[}
\def\PYZrb{]}
\makeatother


    % Exact colors from NB
    \definecolor{incolor}{rgb}{0.0, 0.0, 0.5}
    \definecolor{outcolor}{rgb}{0.545, 0.0, 0.0}



    
    % Prevent overflowing lines due to hard-to-break entities
    \sloppy 
    % Setup hyperref package
    \hypersetup{
      breaklinks=true,  % so long urls are correctly broken across lines
      colorlinks=true,
      urlcolor=urlcolor,
      linkcolor=linkcolor,
      citecolor=citecolor,
      }
    % Slightly bigger margins than the latex defaults
    
    \geometry{verbose,tmargin=1in,bmargin=1in,lmargin=1in,rmargin=1in}
    
    

    \begin{document}
    
    
    \maketitle
    
    

    
    \hypertarget{unit-3---in-class-exercise}{%
\subsubsection{Unit 3 - In-class
exercise}\label{unit-3---in-class-exercise}}

\hypertarget{clinical-trial-simulation}{%
\paragraph{Clinical Trial Simulation}\label{clinical-trial-simulation}}

Suppose that a clinical trial to test a new drug will be conducted on 8
patients, in which the probability of a good response to the drug is
thought to be 0.15. The following code tabulates and graphs the
distribution of the number of good responses based on 1,000 replicates
of the trial for 8 patients.

    \begin{Verbatim}[commandchars=\\\{\}]
{\color{incolor}In [{\color{incolor}9}]:} \PY{c+c1}{\PYZsh{}load necessary packaages}
        \PY{k+kn}{require}\PY{p}{(}mosaic\PY{p}{)}
        \PY{k+kn}{require}\PY{p}{(}dplyr\PY{p}{)}
        \PY{k+kn}{require}\PY{p}{(}ggplot2\PY{p}{)}
        
        \PY{c+c1}{\PYZsh{}set the parameters}
        number.patients \PY{o}{=} \PY{l+m}{8}
        response.prob \PY{o}{=} \PY{l+m}{0.15}
        number.replicates \PY{o}{=} \PY{l+m}{1000}
        
        \PY{c+c1}{\PYZsh{}set the seed}
        \PY{k+kp}{set.seed}\PY{p}{(}\PY{l+m}{2018}\PY{p}{)}
        
        \PY{c+c1}{\PYZsh{}run the simulation}
        number.responses.replicate \PY{o}{\PYZlt{}\PYZhy{}} do\PY{p}{(}number.replicates\PY{p}{)}\PY{o}{*}\PY{p}{\PYZob{}}
            number.responses \PY{o}{=} \PY{k+kp}{sample}\PY{p}{(}\PY{k+kt}{c}\PY{p}{(}\PY{l+m}{0}\PY{p}{,}\PY{l+m}{1}\PY{p}{)}\PY{p}{,} size \PY{o}{=} number.patients\PY{p}{,} replace \PY{o}{=} \PY{k+kc}{TRUE}\PY{p}{,}
                     prob \PY{o}{=} \PY{k+kt}{c}\PY{p}{(}\PY{l+m}{1} \PY{o}{\PYZhy{}} response.prob\PY{p}{,} response.prob\PY{p}{)}\PY{p}{)}
            \PY{k+kt}{data.frame}\PY{p}{(}response.counts\PY{o}{=}\PY{k+kp}{sum}\PY{p}{(}number.responses\PY{p}{)}\PY{p}{)}
        \PY{p}{\PYZcb{}}
        
        \PY{c+c1}{\PYZsh{}create a table of response counts}
        number.responses.replicate \PY{o}{\PYZpc{}\PYZgt{}\PYZpc{}}
            count\PY{p}{(}response.counts\PY{p}{)}
        
        \PY{c+c1}{\PYZsh{}plot the response counts}
        ggplot\PY{p}{(}data\PY{o}{=}number.responses.replicate\PY{p}{,}aes\PY{p}{(}response.counts\PY{p}{)}\PY{p}{)} \PY{o}{+}
            geom\PYZus{}bar\PY{p}{(}alpha\PY{o}{=}\PY{l+m}{.4}\PY{p}{)}
\end{Verbatim}


    \begin{tabular}{r|ll}
 response.counts & n\\
\hline
	 0   & 264\\
	 1   & 410\\
	 2   & 247\\
	 3   &  66\\
	 4   &  12\\
	 5   &   1\\
\end{tabular}


    
    
    
    \begin{center}
    \adjustimage{max size={0.9\linewidth}{0.9\paperheight}}{output_1_2.png}
    \end{center}
    { \hspace*{\fill} \\}
    
    \begin{enumerate}
\def\labelenumi{\alph{enumi})}
\tightlist
\item
  Run the code. From the barplot, describe the distribution.
\end{enumerate}

    \textbf{Solution:}

The distribution is right-skewed; there are very few replicates where
more than 3 patients respond well to the drug. In most of the
replicates, 1 patient out of 8 responds well.

    \begin{enumerate}
\def\labelenumi{\alph{enumi})}
\setcounter{enumi}{1}
\tightlist
\item
  Based on the results of the simulation, estimate the probability that
  0 patients respond well to the new drug.
\end{enumerate}

    \hypertarget{solution}{%
\paragraph{Solution:}\label{solution}}

The probability that 0 patients respond well is \(264/1000 = 0.264\).

    \begin{enumerate}
\def\labelenumi{\alph{enumi})}
\setcounter{enumi}{2}
\tightlist
\item
  Based on the results of the simulation, construct a probability
  distribution for the random variable \(X\), the number of patients who
  respond well to the experimental drug.
\end{enumerate}

    \hypertarget{solution}{%
\paragraph{Solution:}\label{solution}}

    \begin{Verbatim}[commandchars=\\\{\}]
{\color{incolor}In [{\color{incolor}16}]:} number.responses.replicate \PY{o}{\PYZpc{}\PYZgt{}\PYZpc{}}
           group\PYZus{}by\PY{p}{(}response.counts\PY{p}{)} \PY{o}{\PYZpc{}\PYZgt{}\PYZpc{}}
           summarise \PY{p}{(}n \PY{o}{=} n\PY{p}{(}\PY{p}{)}\PY{p}{)} \PY{o}{\PYZpc{}\PYZgt{}\PYZpc{}}
           mutate\PY{p}{(}freq \PY{o}{=} n \PY{o}{/} \PY{k+kp}{sum}\PY{p}{(}n\PY{p}{)}\PY{p}{)}
\end{Verbatim}


    \begin{tabular}{r|lll}
 response.counts & n & freq\\
\hline
	 0     & 264   & 0.264\\
	 1     & 410   & 0.410\\
	 2     & 247   & 0.247\\
	 3     &  66   & 0.066\\
	 4     &  12   & 0.012\\
	 5     &   1   & 0.001\\
\end{tabular}


    
    \begin{longtable}[]{@{}clllllllllll@{}}
\toprule
\endhead
\(x_i\) & 0 & 1 & 2 & 3 & 4 & 5 & 6 & 7 & 8 & Total &\tabularnewline
\(P(X = x_i)\) & 0.264 & 0.410 & 0.247 & 0.066 & 0.012 & 0.001 & 0 & 0 &
0 & = 1.00 &\tabularnewline
\bottomrule
\end{longtable}

    \begin{enumerate}
\def\labelenumi{\alph{enumi})}
\setcounter{enumi}{3}
\tightlist
\item
  What value(s) for response probability would produce a left-skewed
  distribution? what value(s) would produce a symmetric distribution?
\end{enumerate}

    \hypertarget{solution}{%
\paragraph{Solution:}\label{solution}}

Consider the following two examples in which we change the response
probabilities to 0.85 and 0.80.

    \begin{Verbatim}[commandchars=\\\{\}]
{\color{incolor}In [{\color{incolor}17}]:} \PY{c+c1}{\PYZsh{}set the parameters}
         number.patients \PY{o}{=} \PY{l+m}{8}
         response.prob \PY{o}{=} \PY{l+m}{0.85}
         number.replicates \PY{o}{=} \PY{l+m}{1000}
         
         \PY{c+c1}{\PYZsh{}set the seed}
         \PY{k+kp}{set.seed}\PY{p}{(}\PY{l+m}{2018}\PY{p}{)}
         
         \PY{c+c1}{\PYZsh{}run the simulation}
         number.responses.replicate \PY{o}{\PYZlt{}\PYZhy{}} do\PY{p}{(}number.replicates\PY{p}{)}\PY{o}{*}\PY{p}{\PYZob{}}
             number.responses \PY{o}{=} \PY{k+kp}{sample}\PY{p}{(}\PY{k+kt}{c}\PY{p}{(}\PY{l+m}{0}\PY{p}{,}\PY{l+m}{1}\PY{p}{)}\PY{p}{,} size \PY{o}{=} number.patients\PY{p}{,} replace \PY{o}{=} \PY{k+kc}{TRUE}\PY{p}{,}
                      prob \PY{o}{=} \PY{k+kt}{c}\PY{p}{(}\PY{l+m}{1} \PY{o}{\PYZhy{}} response.prob\PY{p}{,} response.prob\PY{p}{)}\PY{p}{)}
             \PY{k+kt}{data.frame}\PY{p}{(}response.counts\PY{o}{=}\PY{k+kp}{sum}\PY{p}{(}number.responses\PY{p}{)}\PY{p}{)}
         \PY{p}{\PYZcb{}}
         
         \PY{c+c1}{\PYZsh{}create a table of response counts}
         number.responses.replicate \PY{o}{\PYZpc{}\PYZgt{}\PYZpc{}}
             count\PY{p}{(}response.counts\PY{p}{)}
         
         \PY{c+c1}{\PYZsh{}plot the response counts}
         ggplot\PY{p}{(}data\PY{o}{=}number.responses.replicate\PY{p}{,}aes\PY{p}{(}response.counts\PY{p}{)}\PY{p}{)} \PY{o}{+}
             geom\PYZus{}bar\PY{p}{(}alpha\PY{o}{=}\PY{l+m}{.4}\PY{p}{)}
\end{Verbatim}


    \begin{tabular}{r|ll}
 response.counts & n\\
\hline
	 3   &   1\\
	 4   &  12\\
	 5   &  66\\
	 6   & 247\\
	 7   & 410\\
	 8   & 264\\
\end{tabular}


    
    
    
    \begin{center}
    \adjustimage{max size={0.9\linewidth}{0.9\paperheight}}{output_12_2.png}
    \end{center}
    { \hspace*{\fill} \\}
    
    \begin{Verbatim}[commandchars=\\\{\}]
{\color{incolor}In [{\color{incolor}18}]:} \PY{c+c1}{\PYZsh{}set the parameters}
         number.patients \PY{o}{=} \PY{l+m}{8}
         response.prob \PY{o}{=} \PY{l+m}{0.50}
         number.replicates \PY{o}{=} \PY{l+m}{1000}
         
         \PY{c+c1}{\PYZsh{}set the seed}
         \PY{k+kp}{set.seed}\PY{p}{(}\PY{l+m}{2018}\PY{p}{)}
         
         \PY{c+c1}{\PYZsh{}run the simulation}
         number.responses.replicate \PY{o}{\PYZlt{}\PYZhy{}} do\PY{p}{(}number.replicates\PY{p}{)}\PY{o}{*}\PY{p}{\PYZob{}}
             number.responses \PY{o}{=} \PY{k+kp}{sample}\PY{p}{(}\PY{k+kt}{c}\PY{p}{(}\PY{l+m}{0}\PY{p}{,}\PY{l+m}{1}\PY{p}{)}\PY{p}{,} size \PY{o}{=} number.patients\PY{p}{,} replace \PY{o}{=} \PY{k+kc}{TRUE}\PY{p}{,}
                      prob \PY{o}{=} \PY{k+kt}{c}\PY{p}{(}\PY{l+m}{1} \PY{o}{\PYZhy{}} response.prob\PY{p}{,} response.prob\PY{p}{)}\PY{p}{)}
             \PY{k+kt}{data.frame}\PY{p}{(}response.counts\PY{o}{=}\PY{k+kp}{sum}\PY{p}{(}number.responses\PY{p}{)}\PY{p}{)}
         \PY{p}{\PYZcb{}}
         
         \PY{c+c1}{\PYZsh{}create a table of response counts}
         number.responses.replicate \PY{o}{\PYZpc{}\PYZgt{}\PYZpc{}}
             count\PY{p}{(}response.counts\PY{p}{)}
         
         \PY{c+c1}{\PYZsh{}plot the response counts}
         ggplot\PY{p}{(}data\PY{o}{=}number.responses.replicate\PY{p}{,}aes\PY{p}{(}response.counts\PY{p}{)}\PY{p}{)} \PY{o}{+}
             geom\PYZus{}bar\PY{p}{(}alpha\PY{o}{=}\PY{l+m}{.4}\PY{p}{)}
\end{Verbatim}


    \begin{tabular}{r|ll}
 response.counts & n\\
\hline
	 0   &   2\\
	 1   &  30\\
	 2   & 100\\
	 3   & 207\\
	 4   & 272\\
	 5   & 234\\
	 6   & 121\\
	 7   &  30\\
	 8   &   4\\
\end{tabular}


    
    
    
    \begin{center}
    \adjustimage{max size={0.9\linewidth}{0.9\paperheight}}{output_13_2.png}
    \end{center}
    { \hspace*{\fill} \\}
    
    A response probability close to 1, such as 0.85, produces a left-skewed
distribution; in this case, we would expect almost all of the 8 patients
in a trial to respond well to the drug.

A response probability near 0.5 produces a symmetric graph. In this
case, we would typically expect half the patients out of 8 to respond
well to the drug.

    \begin{enumerate}
\def\labelenumi{\alph{enumi})}
\setcounter{enumi}{4}
\tightlist
\item
  Calculate \(E(X)\), where \(X\) is the number of patients who respond
  well to the experimental drug.
\end{enumerate}

    \hypertarget{solution}{%
\paragraph{Solution:}\label{solution}}

\begin{align*}
E(X)    &= \sum_{i=1}^{k}x_iP(X=x_i) \notag \\
    &= (x_1)P(X = x_1) + (x_2)P(X = x_2) + \cdots + (x_k)P(X = x_k) \\
    &= (0)(0.264) + (1)(0.410) + (2)(0.247) + (3)(0.066) \\
    & \qquad + (4)(0.012) + (5)(0.001) + (6)(0) + (7)(0) + (8)(0) \\
    &= 1.155
\end{align*}

    \begin{enumerate}
\def\labelenumi{\alph{enumi})}
\setcounter{enumi}{5}
\tightlist
\item
  Calculate \(SD(X)\), where \(X\) is the number of patients who respond
  well to the experimental drug. It is sufficient to write the answer in
  an unsimplified form where only simple arithmetic is necessary to
  reach the final value.
\end{enumerate}

    \hypertarget{solution}{%
\paragraph{Solution:}\label{solution}}

\begin{align*}
Var(X)  &= \sum_{j=1}^{k} (x_j - \mu)^2 P(X=x_j) \\
  &= (x_1-\mu)^2 P(X=x_1) + \cdots+ (x_k-\mu)^2 P(X=x_k) \\
    &= (0 - 1.155)^{2}(0.264) + (1 - 1.155)^{2}(0.410) + (2 - 1.155)^{2}(0.247) + (3 - 1.155)^{2}(0.066) \\
      & \qquad + (4 - 1.155)^{2}(0.012) + (5 - 1.155)^{2}(0.001) + (6 - 1.155)^{2}(0) + (7 - 1.155)^{2}(0) \\
      & \qquad (8 - 1.155)^{2}(0) \\
    &= 0.875
\end{align*}

\(SD(X) = \sqrt{Var(X)} = 0.935\).


    % Add a bibliography block to the postdoc
    
    
    
    \end{document}
